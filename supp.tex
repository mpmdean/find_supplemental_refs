%%%%%%%%%%%%%%%%%%%%%%%%%%%%%%%%%%%%%%%%%%%%
\documentclass[aps,prx,showpacs,floatfix,onecolumn,superscriptaddress,longbibliography,notitlepage]{revtex4-1}

\usepackage{float}
\usepackage{color}
\usepackage{bm}
\usepackage{upgreek}
\usepackage{todonotes}
\usepackage{verbatim}
\usepackage{soul}
\usepackage{glossaries}
\usepackage{sidecap}
\usepackage{hyperref}% add hypertext capabilities
\hypersetup{
    colorlinks=true,
    linkcolor=blue,
    filecolor=magenta,      
    urlcolor=red,
    citecolor=blue,
}

\def\Q{\ensuremath{\bm{Q}}}
\def\LSCO{L\MakeLowercase{a}$_{2-x}$S\MakeLowercase{r}$_x$C\MakeLowercase{u}O$_4$}

\begin{document}

\title{Supplemental Material: Template for Physical Review Letters}

\renewcommand{\thepage}{S\arabic{page}} 
\renewcommand{\thesection}{S\arabic{section}}  
\renewcommand{\thetable}{S\arabic{table}}  
\renewcommand{\thefigure}{S\arabic{figure}}

\date{\today}

\maketitle

This document provides a suggested structure for the supplemental materials, which comprises a introductory paragraph that defines the contents of the document followed by separate sections. It can be useful to number  the sections and to prefix all numbers with ``S'' so that the elements of the main and supplemental materials can be easily distinguished.

\section{Growth of \texorpdfstring{L\MakeLowercase{a}$_{2-x}$S\MakeLowercase{r}$_x$C\MakeLowercase{u}O$_4$}{LSCO}}
Single crystals of \LSCO{} were grown using the flux method as described in Ref.~\cite{Cao2020Quantum}. A measurement of \LSCO{} is shown in Fig.~\ref{fig:supp_fig_growth}. In the section heading, there we needed to include some tricks to ensure lowercase and to escape some warnings.


\begin{figure*}
\includegraphics{supp_fig_growth.pdf}
\caption{Caption.}
\label{fig:supp_fig_growth}
\end{figure*}

\clearpage
\bibliography{refs}
\end{document}

